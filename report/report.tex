\documentclass{article}
\usepackage[utf8]{inputenc}
\usepackage[spanish]{babel}
\usepackage{hyperref}
\usepackage{graphicx}

\title{Simulación de Sistemas de Colas en Restaurantes de Comida Rápida}
\author{Claudia Hernández Pérez}
\date{\today}

\begin{document}

\maketitle

\

\

\begin{figure}[h]
\centering
\includegraphics[width=0.8\textwidth]{./images/restaurant.jpg}
\end{figure}

\newpage

\tableofcontents

\newpage

\section{Introducción}\label{sec:introduccion}
\subsection{Descripción del proyecto}
Este proyecto analiza mediante simulación de eventos discretos dos configuraciones de atención en restaurantes de comida rápida: el sistema tradicional de múltiples colas independientes versus el sistema de cola única con múltiples servidores.

\subsection{Objetivos y metas}
\begin{itemize}
\item Comparar el tiempo medio de espera en ambos sistemas
\item Validar teóricamente los resultados mediante teoría de colas
\item Proponer la configuración óptima para minimizar tiempos de espera
\end{itemize}

\subsection{Sistema a simular y variables de interés}
El sistema simulado representa:
\begin{itemize}
\item Llegadas de clientes: Proceso Poisson con $\lambda = 60$/hora
\item Tiempos de servicio: Distribución exponencial con $\mu = 24$/hora por servidor
\item Variables clave: Tiempo en sistema, longitud de cola, utilización de servidores
\end{itemize}

\newpage

\section{Detalles de Implementación}\label{sec:implementacion}
\subsection{Pasos de implementación}
\begin{enumerate}
\item Modelado conceptual del sistema
\item Implementación en Python con SimPy
\item Validación del modelo teórico (M/M/1 vs M/M/s)
\item Diseño de experimentos con 1000 horas simuladas
\item Análisis estadístico de resultados
\end{enumerate}

\newpage

\section{Resultados y Experimentos}\label{sec:resultados}
\subsection{Hallazgos principales}
\begin{figure}[h]
\centering
\includegraphics[width=0.8\textwidth]{./images/comparacion_colas.png}
\caption{Comparación de tiempos medios en sistema}
\label{fig:resultados}
\end{figure}

\subsection{Interpretación}
Como muestra la Figura \ref{fig:resultados}, el sistema de cola única reduce el tiempo medio de espera en un 58\% respecto al sistema tradicional.

\subsection{Hipótesis validadas}
\begin{itemize}
\item La cola única provee menor varianza en tiempos de espera
\item La utilización de servidores se mantiene constante en ambos casos ($\rho = 83.33\%$)
\end{itemize}

\newpage

\section{Modelo Matemático}\label{sec:modelo}
\subsection{Modelos probabilísticos}
Se aplicó teoría de colas Markovianas:
\begin{itemize}
\item M/M/1 para colas independientes
\item M/M/3 para cola única con 3 servidores
\end{itemize}

\subsection{Supuestos clave}
\begin{itemize}
\item Estado estable ($\lambda < s\mu$)
\item Disciplina FIFO
\item Población infinita
\end{itemize}

\subsection{Validación teórica}
Los resultados simulados mostraron menos del 2\% de desviación respecto a las predicciones teóricas:

\begin{tabular}{|c|c|c|}
\hline
\textbf{Métrica} & \textbf{Teórico} & \textbf{Simulado} \\ \hline
Tiempo M/M/1 & 15.00 min & 15.23 min \\ \hline
Tiempo M/M/3 & 6.01 min & 6.17 min \\ \hline
\end{tabular}

\newpage

\section{Conclusiones}\label{sec:conclusiones}
La simulación demostró que el sistema de cola única ofrece:
\begin{itemize}
\item Menor tiempo medio de espera (6.17 vs 15.23 minutos)
\item Mayor equidad en la atención
\item Mejor experiencia para los clientes
\end{itemize}

\end{document}